% Note: this file can be compiled on its own, but is also included by
% diss.tex (using the docmute.sty package to ignore the preamble)
\documentclass[12pt,a4paper,twoside]{article}
\usepackage[pdfborder={0 0 0}]{hyperref}
\usepackage[margin=25mm]{geometry}
\usepackage{graphicx}
\begin{document}

\vfil

\centerline{\Large Computer Science Project Proposal}
\vspace{0.4in}
\centerline{\Large Encrypted Keyword Search Using}
\vspace{0.05in}
\centerline{\Large Path ORAM on MirageOS}
\vspace{0.4in}
\centerline{\large R. Horlick, Homerton College}
\vspace{0.3in}
\centerline{\large Originator: Dr N. Sultana}
\vspace{0.3in}
\centerline{\large 23 October 2015}

\vfil


\noindent
{\bf Project Supervisors:} Dr N. Sultana \& Dr R. M. Mortier
\vspace{0.2in}

\noindent
{\bf Director of Studies:} Dr B. Roman
\vspace{0.2in}
\noindent
 
\noindent
{\bf Project Overseers:} Dr M. G. Kuhn \& Prof P. M. Sewell


% Main document

\section*{Introduction and Description of the Work}

% Set the scene - change this so that it mentions that what we want is encrypted search

As the cost of large-scale cloud storage decreases and the rate of data production grows, more and more sensitive data is being stored in the cloud. We, of course, want to encrypt our data, to ward off prying eyes, but this comes at a cost. We can no longer selectively retrieve parts of the data at will. We need some method of searching over encrypted data to find the parts we are interested in.

% Set up the threat model for "honest but curious" adversary

So let us say that Alice has a set of documents that she wants to store on an untrusted server, run by Bob. We'll first assume that Bob is ``honest, but curious'', that is, he will attempt to gather all knowledge that he can without deviating from the protocol. Alice wants to store her documents encrypted, but also wants to search over them without Bob being able to learn either the keywords she is searching for, or the results of any query, the documents that contain the keyword. In order to enable efficient search over the document, Alice also stores an encrypted index on the server.

There are a number of schemes in the literature that use symmetric encryption techniques to build a searchable encryption scheme. They rely on the use of a trapdoor generating function, that allows Bob to search over the encrypted index and respond to Alice with the matching line from the encrypted index. Then Alice requests the relevant documents from Bob.

The problem is that these all leak the access pattern, so Bob knows which documents matched any query, even if he doesn't know what they matched. It turns out that this pattern of access can leak large amounts of information. In a study [1] on an encrypted email repository, up to 80\% of plaintext search queries could be inferred from the access pattern alone! So clearly this a leak worth plugging, but how can we do it?

% Describe ORAM and mention Path ORAM

One solution to our problem is to use Oblivious Random Access Memory (ORAM), a cryptographic primitive that hides data access patterns. That is, we turn Bob's server into a block device and we attempt to maintain the property that any two sequences of accesses of the form $(operation,address,data)$, that are the same length, have computationally indistinguishable physical access patterns. Bob should have no way of learning what $address$ we are really accessing, and therefore will never know which documents matched a given search query.

A trivial ORAM algorithm operates by scanning over the whole ORAM and reading/updating only the relevant block, but this has $O(N)$ bandwidth cost, where $N$ is the number of blocks, which is highly impractical for large-scale storage. Luckily, much better algorithms have been proposed. We choose to focus on Path ORAM, because it has only $O(\log N)$ bandwidth cost in the worst case if $B = \Omega(\log^2 N)$, as well as being incredibly simple conceptually.

% Extend the threat model to incorporate malicious adversary and describe integrity protection techniques

Now let's assume that Bob has become malicious, and is modifying our encrypted data. In order to combat this, we can provide integrity verification by treating the ORAM as a Merkle tree, but with data in every node. The details of this scheme are outlined below after Path ORAM has been described further.

So the project is a searchable encrypted object store, with integrity verification. It will provide a simple, name-value pair API, that allows more complex filesystems to be built on top of it. A block diagram of the system is shown in Figure \ref{miragestack}.

\begin{figure}
\setlength{\unitlength}{0.75mm}
\begin{center}
\begin{picture}(170,70)
\put(50,60){\framebox(50,10){Object Store}}

\put(100,65){\vector(1,0){20}}
\put(120,65){\vector(-1,0){20}}

\put(120,60){\framebox(50,10){Indexed Search}}

\multiput(44,55)(4,0){32}{\line(1,0){2}}
\put(0,54){BLOCK Interface}
\put(75,60){\vector(0,-1){10}}
\put(75,50){\vector(0,1){10}}

\put(145,45){\vector(0,1){15}}
\put(145,45){\vector(-1,0){45}}

\put(50,40){\framebox(50,10){ORAM}}

\multiput(44,35)(4,0){32}{\line(1,0){2}}
\put(0,34){BLOCK Interface}
\put(75,40){\vector(0,-1){10}}
\put(75,30){\vector(0,1){10}}

\put(50,20){\framebox(50,10){Encryption}}

\multiput(44,15)(4,0){32}{\line(1,0){2}}
\put(0,14){BLOCK Interface}
\put(75,20){\vector(0,-1){10}}
\put(75,10){\vector(0,1){10}}

\put(50,0){\framebox(50,10){BLOCK}}

\end{picture}
\end{center}
\caption{The Application Stack: We can use any underlying BLOCK implementation and we can add/remove ORAM, Encryption or Search modules as we please}
\label{miragestack}
\end{figure}

\section*{Starting Point}

% Explain Mirage (further explanation of Unikernel OS required) and justify its use

MirageOS is a framework, that pulls together a number of libraries and syntax extensions, to provide a lightweight unikernel operating system, that is designed to run on the Xen hypervisor. A unikernel operating system is a single-address space machine image, customised to provide the minimum set of features to run an application. It provides a command line tool for generating the main file, that links together implementations of various parts of the system, and passes them to the unikernel. There are a number of module signatures that define the operation of devices, such as CONSOLE for consoles, ETHIF for ethernet, and most importantly for us BLOCK, for block devices.

I have chosen to use Mirage for a number of reasons. Firstly, it is lightweight and designed to be run in the cloud, meaning that simple cloud services can be built on top of it that fully leverage the ORAM. Secondly, it is written in OCaml, meaning that I can take full advantage of static typing and a rich module system. 

This will allow me to write my implementation of ORAM and Encryption as a pair of functors that take an implementation of Mirage's BLOCK interface and create new BLOCK implementations, augmented with new features. This means that we can add and remove ORAM and Encryption as we like and the Object Store remains agnostic. This is shown in Figure \ref{miragestack}. It also means that we could use any underlying implementation of the BLOCK interface and that it would plug seamlessly into existing programs. There are currently two implementations of the BLOCK interface, one for Unix and one for Xen, and I would like to support both. This abstraction also allows for the use of cloud storage, implemented as a mapping between the BLOCK interface and a cloud provider's RESTful API.

Other OCaml libraries that will be of most use to me include nocrypto, which provides a wide variety of cryptographic tools, Jane Street's Core library, which standardises and optimises many of OCaml's core modules, and LWT, a lightweight cooperative threading library that is used throughout Mirage.

\section*{Substance and Structure of the Project}

\subsection*{Substance}

% Talk about the relative priorities of sections of the program

	% Main focus is ORAM, then we build object store on top of that, finally we implement the actual search module, but make it clear that, that is not the focus
	
	The main focus of this project is the implementation and evaluation of the Path ORAM protocol. Encrypted search is our target domain and as such will be an integral part of the project, but it is the performance and security properties that Path ORAM provides that we are really interested in.

% Describe Path ORAM protocol more clearly

The Path ORAM protocol has three main components: a binary tree, a stash and a position map. The binary tree is the main storage space. Every node in the tree is a bucket, which can contain up to $Z$ blocks. The tree has height $L$, where the tree of height 0 consists only of the root node, and the leaves are at level $L$. The stash is temporary client-side storage, consisting only of a set of blocks waiting to be put back into the tree. The position map associates, with each block ID, an integer between 0 and $2^L-1$. The invariant that the Path ORAM algorithm maintains is that if the position of a block $x$ is $p$, then $x$ is either in some bucket along the path from the root node to the $p^{th}$ leaf, or in the stash. On every access to the tree, a whole path is read into the stash, the accessed block is assigned a new random position and then as many blocks as possible are written back into the same path. The assignment to a random position means that, in any two access to the same block, the paths that are read are statistically independent.

% Describe recursive path oram protocol

We can extend the basic path ORAM algorithm with recursion. That is, calling the data ORAM $ORAM_0$, we store the position map of $ORAM_0$ in a smaller ORAM, $ORAM_1$, and the position map for this in an even smaller ORAM, $ORAM_2$. We can do this until we have a sufficiently small position map on the client. Supposing that we store $\chi$ leaf addresses in each PosMap ORAM, the position for a data block with address $a_0$ is at $a_1 = a_0 / \chi$ in $ORAM_1$, and in general $a_n = a_0 / \chi^n$ for the address in $ORAM_n$. 

% Mention that we want to be able to disconnect from the block device, so we'll need to be able to store the client state on the ORAM as well

There is actually an issue with using Path ORAM in the context of MirageOS and cloud storage. If the Mirage instance crashes, then we lose the client-side state. With no position map, the ORAM becomes useless. To remedy this, we would have to read the entire contents out in one go and then reinsert it, resulting in a large overhead. There are two solutions to this problem: store the client-side state in persistent storage on the client, or upload the state to the server after every access. The second option is preferable, because it separates the ORAM implementation from the client machine. The client-side state is actually $O(\log N)$, so we should be able to store it on the server without increasing our complexity bounds. 

% Talk about adding integrity verification of the ORAM

As mentioned above, we can add integrity verification to Path ORAM by treating it as a Merkle tree. Each node will store a hash of the form $H = (b_1||\cdots||b_n||h_1||h_2)$, where $b_n$ is the $n^{th}$ block stored in the node, and $h_1$ and $h_2$ are the hashes of the left and right children. We always read and write the whole path at a time, so for the read or write of any single node we only have to read or write two hashes. For instance, on write, we calculate the hash of the leaf node, which is then available for calculating the next level hash. So we only have to read the hash of the sibling of the leaf. This pattern is the same all the way up to the root of the tree.

% Discuss the search mechanism

In order to perform searches over our data, we will store along with it an encrypted inverted index. This is a data structure that, for any keyword, list the documents that contain it. The search module will build the index from the object store and then store the index using the object store. It will provide a search function that, given a keyword, will perform a simple scan over the inverted index and return the identifiers of documents that match it.

% Discuss evaluation techniques

Evaluation will consist of a few different types of testing. 

We need to test for functionality. Does the ORAM successfully write data and read it back out? Does the search function return documents correctly? This will consist of fairly trivial tests, writing objects to and from the block device and searching over them.

We then want to evaluate performance. What is the overhead when we add the ORAM functor? How do recursion and statelessness further affect this? Does this correspond to the theoretical values from the literature?

Finally we want to test the security properties of the project. Is there any statistical correlation between access patterns? Do we provide adequate integrity verification? What, if anything, can be inferred about the search queries?

\subsection*{Structure}

The project breaks down into the following sub-projects:

\begin{enumerate}

\item Familiarising myself with OCaml, MirageOS and related libraries

\item Implementing the basic Path ORAM functor and testing that it works in place of existing BLOCK device implementations

\item Implementing the Object Store, testing this and further testing ORAM using it

\item Adding recursion and statelessness to ORAM 

\item Implementing and testing the search module

\item Adding the encryption layer

\item Creation of a suite of tests and experiments to evaluate the performance and security properties of each individual component and of the system as a whole

\item Writing the dissertation

\end{enumerate}

\section*{Success Criteria for the Main Result}

\begin{itemize}
	\item To demonstrate, through well chosen examples, that I have implemented a functionally correct Path ORAM functor with search capabilities
	\item To demonstrate, through well chosen examples, that the implementation has the expected security properties, i.e. keeps access patterns hidden
\end{itemize}


\section*{Possible Extensions}

There are a number of ways that this project could be extended. By the nature of the modular design, we can perform optimisations at any layer of the system. There have been a large number of optimisations to Path ORAM proposed in the literature, so if I achieve the goals of my main project, including evaluation, ahead of schedule I will examine these and potentially implement some of them.

In particular for Path ORAM, recursion does add an overhead, but we can reduce this overhead by exploiting locality. Assuming that programs will access adjacent data blocks, we can cache PosMap blocks in a PosMap Lookaside Buffer, so that if all $\chi$ data blocks that are referenced in a PosMap block are accessed in turn we only need to do the recursion once. Doing this na\"ively, however, breaks security, because we are revealing information through the cache hit pattern. To avoid this we use Unified ORAM, which combines all of the recursive ORAMs into a single logical tree. We then use the address space to separate the levels of recursion, so addresses 1 to $N$ are for data blocks, $N + 1$ to $N + (N / \chi)$ for $ORAM_1$ and so on. Now all accesses occur in the same tree, and the security of Path ORAM keeps the cache miss pattern hidden.

Another optimisation compresses the PosMaps, reducing the number of levels of recursion required to achieve the desired client side storage, resulting in an asymptotic bandwidth complexity decrease for ORAM with small block size.

The last two optimisations were originally designed and tested in a secure processor setting, so their application to the cloud storage setting is novel.

Another area that could be addressed is the limitation of Path ORAM (and other tree based ORAMs) to fixed-sized trees. We either need to know our storage requirements before setting up the ORAM, potentially wasting resources, or resize them in the na\"ive way as storage requirements increase. It would be interesting, although open-ended, to explore the possibility of making Path ORAM resizable.

\section*{Timetable: Work plan and Milestones}

Planned starting date is 16/10/2015.

\begin{enumerate}

\item {\bf 16/10/15 -- 26/10/15} Familiarise myself with relevant Mirage libraries. Implement basic Path ORAM functor.

\item {\bf 27/10/15 -- 09/11/15} Implement basic test harness. Start implementation of object store.

\item {\bf 10/11/15 -- 23/11/15} Finish object store and use it to build more complex tests of the ORAM.

\item {\bf 24/11/15 -- 04/12/15} Add recursion and statelessness to the ORAM.

\item {\bf 05/12/15 -- 18/12/15} Write up implementation section of the dissertation for all parts completed so far.

\item {\bf 18/12/15 -- 31/12/15} Write the search module and design tests for it.

\item {\bf 01/01/16 -- 08/01/16} Write implementation section for the search module.

\item {\bf 09/01/16 -- 29/01/16} Evaluate the project in its current state, achieving an acceptably complete project. Write the progress report.

\item {\bf 30/01/16 -- 08/02/16} Write up the evaluation of the project so far.

\item {\bf 09/02/16 -- 21/02/16} Incorporate encryption model and perform further evaluation using this.

\item {\bf 22/02/16 -- 06/03/16} Submit first draft to supervisors for feedback and modify based on feedback.

\item {\bf 07/03/16 -- 11/03/16} Perform further evaluation and refinement as necessary

\item {\bf 12/03/16 -- 25/03/16} Write final draft of dissertation and then leave it until submission time, in order to focus on revision.

\item {\bf 01/05/16 -- 13/05/16} Reread and make any final edits and then submit.

\end{enumerate}

\section*{Resources Required}

\begin{itemize}

	\item My own laptop for implementation and testing

	\item My own external hard disk for backups

	\item GitHub for version control and backup storage

	\item MirageOS libraries as a basis for the project

\end{itemize}

\end{document}
